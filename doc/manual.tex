\documentclass[a4paper]{article}

\usepackage[english]{babel} %language selection
\selectlanguage{english}

\usepackage{pdfpages}
\usepackage{amssymb}
\usepackage[utf8]{inputenc}
\usepackage[colorlinks]{hyperref}
\usepackage{cleveref}
\usepackage{cite}
\usepackage{verbatim}
\usepackage{tikz}
\usetikzlibrary{backgrounds,calc}

\newcommand{\memorax}{\textsc{Memorax}}
\newcommand{\rmm}{\textsc{Rmm}}
\newcommand{\binary}{memorax} % Name of the command line executable
\newcommand{\gui}{memorax-gui} % Name of the GUI executable

\title{Memorax\\\small{User Manual}}
\author{Carl Leonardsson}

\begin{document}

\maketitle

\pagebreak

\begin{verbatim}
Copyright (C)  2012  Carl Leonardsson
Permission is granted to copy, distribute and/or modify this document
under the terms of the GNU Free Documentation License, Version 1.3
or any later version published by the Free Software Foundation;
with no Invariant Sections, no Front-Cover Texts, and no Back-Cover Texts.
A copy of the license is included in the section entitled ``GNU
Free Documentation License''.
\end{verbatim}

\pagebreak

\tableofcontents

\section{Introduction}

\memorax\ is a tool for verification of safety properties in programs
running under relaxed memory models, and for automatic inference of
memory fences that are necessary in order to guarantee satisfaction of
those safety properties.

The verification is achieved through state space exploration and
specifically control state reachability analysis. Inference of fences
is achieved by repeated reachability analysis according to the method
described in \cite{AACLR12}. 

Programs running under relaxed memory models tend to have an infinite
state space, even in the cases where the same program would have a
finite state space if executed under SC. For this reason, explicit
state space analysis is impossible. The problem can be handled by
using under approximation, over approximation or by using exact,
infinite state analysis methods such as the well-quasi ordering
method \cite{ACJT96}. \memorax\ is built to accommodate different
approaches. 

Currently \memorax\ supports two approaches for verification/fence
insertion of programs executed under the TSO memory model. Details are
given in \cref{sec:abstractions}.

This manual will assume basic knowledge about memory models. An
introduction to the topic can be found in \cite{adve-gharachorloo-96}.

\section{Contact / Bug Report}

Feedback, questions or bug reports should be directed to Carl
Leonardsson ({\tt carl.leonardsson@it.uu.se}).

\section{Installation}

\subsection{Requirements}

\begin{enumerate}

  \item A C++ compiler supporting C++11. For example g++ version 4.6 or
     higher.

  \item In order to run the graphical interface, python is required at a
     version of 2.6 or higher installed with tcl/tk of version 8.4 or
     higher.

  \item For predicate abstraction the MathSAT SMT solver as well as the
     library gmpxx are required. \memorax\ supports MathSAT 4 and 
     MathSAT 5. MathSAT 4 is recommended. \memorax\ can be compiled without
     MathSAT and gmpxx, but will then not support predicate
     abstraction.

  \item To be able to graphically draw automata, Graphviz is required.
\end{enumerate}

\subsection{Basic Installation}

   In the simplest case, \memorax\ can be installed with the following
   commands:

\begin{verbatim}
$ tar xvf memorax-<version>.tar.gz
$ cd memorax-<version>
$ ./configure
$ make
$ make install
\end{verbatim}

\subsection{Installation Options}

   The configure script is built with GNU autotools, and should accept
   the usual options and environment variables. This section outlines
   some of the typical use cases.

\paragraph{Changing Installation Directory}

   The command '\verb+make install+' will install \memorax, its graphical
   interface and its documentation in the directories which are
   standard on your system. To override this behaviour add the switch
   \verb+--prefix+ to the '\verb+./configure+' command:

\begin{verbatim}
$ ./configure --prefix=/your/desired/install/path
\end{verbatim}

\paragraph{Compiling with Predicate Abstraction Support}

   To support predicate abstraction, \memorax\ must be compiled with
   MathSAT and gmpxx. Their header files and shared libraries must
   reside where they can be found by the compilation. If they are
   installed in non-standard locations, then the compilation can be
   directed to their location by appropriately specifying
   \verb+CXXFLAGS+ and \verb+LDFLAGS+ when invoking the
   '\verb+./configure+' command:

\begin{verbatim}
$ ./configure CXXFLAGS='-I/path/to/mathsat/include' \
              LDFLAGS='-L/path/to/mathsat/lib'
\end{verbatim}

   If MathSAT and/or gmpxx are not found by the configure script, then
   \memorax\ will be installed without support for predicate abstraction.

\paragraph{Specifying Compiler}

   When the configure script is invoked, it will by GNU autotools
   magic determine which C++ compiler will be used during
   compilation. In case e.g. your default compiler does not support
   C++11, but you have the compiler g++-4.6 installed at a
   non-standard location you may want to override this. In order to do
   so, specify the path to g++-4.6 in CXX when invoking the
   '\verb+./configure+' command:

\begin{verbatim}
$ ./configure CXX='/path/to/g++-4.6'
\end{verbatim}

\subsection{Troubleshooting}

\paragraph{MSatFailure}

   In case you get the following error message when trying to use the
   PB abstraction:

\begin{verbatim}
Error: MSatFailure: Program is not compiled with MathSAT.
\end{verbatim}

   In order to use predicate abstraction, (e.g. the PB abstraction)
   \memorax\ needs to be compiled with MathSAT. To solve the problem,
   install MathSAT on your system and then reinstall \memorax. In case
   the installation fails to find MathSAT (see the output from the
   configure script), then try the instructions in the paragraph
   "Compiling with Predicate Abstraction Support" above.

\section{Usage}\label{sec:usage}

\memorax\ provides a command line interface, as well as a graphical
interface. It has three main modes of operation (henceforth
``\emph{commands}''): reachability checking, automatic fence inference
and graphical representation of programs as automata. In all three
modes \memorax\ works on parallel programs given in the \rmm\ language
(See \cref{sec:rmm:language}.).

The algorithms used for reachability and automatic fence inference
depend on which \emph{abstraction} is selected. An abstraction defines
what a configuration looks like, what are the semantics of the
analysed program and how the reachability analysis works. Abstractions
can be over or under approximations of the semantics given in
\cref{sec:semantics}. They can alternatively be exact, or even an
approximation that is neither an over approximation nor an under
approximation. \memorax\ currently supports two abstractions: SB and
PB. See \cref{sec:abstractions} for details.

\paragraph{Reachability checking}
 In this mode, \memorax\ will attempt to determine whether or not
 certain (``\emph{forbidden}'') configurations are reachable when the
 \rmm\ program is executed. The forbidden configurations are specified
 in the \rmm\ program as combinations of control states; one for each
 process. Any configuration where the processes are each in a control
 state such that they together satisfy such a combination is
 considered forbidden. 

 The reachability is determined by some reachability analysis, which
 depends on what abstraction is chosen.

\paragraph{Automatic fence inference}
In this mode, \memorax\ will perform repeated reachability checks,
while gradually adding memory fences that turn out to be necessary in
order to guarantee the non-reachability of the forbidden control
states. \memorax\ will report a collection $C$ of sets $S$ of memory
fences such that for every set $S$, the memory fences in $S$ are
sufficient to guarantee the non-reachability of the forbidden control
states. Furthermore each fence in $S$ is necessary, in the sense that
adding all fences except one from $S$ to the program, is insufficient
to guarantee non-reachability. Here, ``reachability'' should be
interpreted as reachability according to the given abstraction. Thus,
an over approximating abstraction may report more fences than are
actually necessary under the actual memory model, and an under
approximating abstraction may fail to report fences that are actually
necessary. The abstractions SB and PB (when run without a bound on the
number of refinements) guarantee that the reported fences are both
necessary and sufficient.

For the TSO memory model, reported fences are identified with write
instructions. Adding the fence to the \rmm\ program corresponds to
making that write instruction into a locked write instruction. In the
actual machine code/assembly code that implements the program runnable
on real hardware, this corresponds to adding memory fence immediately
after the writing instruction. On x86, one can alternatively change
the writing instruction into a LOCK'd version of the same instruction.

\paragraph{Graphical representation of \rmm\ programs}
In this mode, \memorax\ produces a PDF file containing a graphical
representation of automata corresponding to the given \rmm\ 
program. There will be one automaton per process in the program. 

\subsection{Using the Command Line Interface}

A call to the command line interface is on the following form:

\vspace{15pt}

\noindent
{\tt \binary{} [command] [options] [program]}

\vspace{15pt}

The {\tt [command]} part indicates the mode of operation. It should be
given as one of {\tt reach} (indicating reachability analysis), {\tt
  fencins} (indicating automatic fence inference) and {\tt dotify}
(indicating graphical representation of the \rmm\ program).

The {\tt [options]} part is optional and gives details about how the
command should be executed. Accepted options are listed and explained
below.

The {\tt [program]} part should be the path to a text file containing
an \rmm\ program. The program path can be left out of the command line
invocation, in which case \memorax\ will expect the \rmm\ program via
the standard input.

\newcommand{\explainrff}{Converting an \rmm\ program to register free
  form, means to rewrite it such that the values of the registers are
  encoded in the control states, and all registers are replaced by the
  corresponding integer literals wherever they occur in
  instructions. This conversion is possible when all registers in the
  program have finite domains. Converting a program to register free
  form may be beneficial for analysis time, in particular when using
  the SB abstraction.}

\paragraph{Options:}

\begin{itemize}
\item {\tt -o <filename>} or {\tt --output <filename>}\\ Write output
  to {\tt <filename>}. This option is used to specify the desired path
  of the PDF file produced by the {\tt dotify} command.

\item {\tt -a <abstraction>} or {\tt --abstraction <abstraction>}
  \\ Use abstraction {\tt <abstraction>}. The abstraction should be
  one of {\tt pb} and {\tt sb}. If no abstraction is specified, then
  \memorax\ will default to using the SB abstraction.

\item {\tt -k <int>}\\ Use {\tt k} as buffer bound. The TSO buffers in
  the PB abstraction will not be allowed to grow larger than this many
  elements.

\item {\tt --cegar}\\ Use CEGAR refinement in reachability
  analysis. CEGAR can be used with the PB abstraction, and will refine
  the abstraction by gradually, and as necessary, using additional
  predicates in the predicate abstraction, and a larger bound on the
  length of the TSO buffers.

\item {\tt --max-refinements <int>}\\ Perform at most {\tt <int>} many
  refinements in the CEGAR loop. If more refinements are necessary,
  then \memorax\ will terminate with an error message.

\item {\tt -v} or {\tt --verbose}\\
  Print output verbosely.
\item {\tt -vv} or {\tt --very-verbose}\\
  Print output very verbosely.
\item {\tt -vvv} or {\tt --very-very-verbose}\\
  Print output very very verbosely.
\item {\tt -o1} or {\tt --only-one}\\
  During fence insertion, stop searching after finding one sufficient, minimal fence set.
\item {\tt --rff}\\
  Convert machine to \emph{register free form}
  before using it. \explainrff
\end{itemize}

\subsection{Using the Graphical Interface}

The graphical interface is a python script using tcl/tk, running on
top of the command line interface. It runs on top of the command line
interface, provides the CLI with appropriate switches and performs
some interpretation of the output from the CLI. The graphical
interface is installed as {\tt \gui{}}.

The GUI window contains, from top to bottom:

\begin{itemize}
\item A menu bar, allowing to load and save \rmm\ programs and output,
  and to configure the behaviour of the GUI.
\item A \emph{command area} containing a number of buttons, check buttons etc.
\item A \emph{code area} where an \rmm\ program may be loaded, edited
  and saved. All commands executed with the GUI will act on the
  program displayed here.
\item An \emph{output area} where text output from the underlying CLI
  will be displayed. The output is divided into two different
  consoles: ``Output'' and ``Error''.
\end{itemize}

At the top of the command area are radio buttons allowing the user to
chose a command: ``Reachability'' (indicating reachability analysis),
``Fence insertion'' (indicating automatic fence inference) and ``Draw
automata'' (indicating graphical representation of the \rmm\ program).

For all commands, the user may specify a level of output verbosity
ranging from ``Only Results'' (least output) to ``Extreme'' (most
output). For most users and use cases, one of the levels ``Only
Results'' and ``Messages'' is probably the most suitable.

Execution of the selected command is started with the button
``Run''. A running execution can be interrupted with the button
``Break'' (shortcut Ctrl-C). While the underlying tool is running, it
will output text to the text fields ``Output'' and ``Error'' at the
bottom of the GUI window.

For commands ``Reachability'' and ``Fence Insertion'', the user may
chose what abstraction should be used: SB or PB. (See
\cref{sec:abstractions}.) Whether CEGAR should be used for
automatically refining the PB abstraction. Also the user may chose to
convert the \rmm\ program into \emph{register free form} before
analysing it. \explainrff

When the command ``Fence Insertion'' finds a set of fences, they are
indicated in the output by textual representation of the writing
transitions in the \rmm\ automata which should be changed into locked
writes. Mouse-over will highlight the corresponding lines of code in
the code area, and clicking them will center the code area over those
lines.

For the command ``Draw Automata'', the user should specify a path,
where a PDF file displaying the \rmm\ program will be created. When the
``Draw Automata'' command is used, the PDF file will be immediately
displayed, provided that the user has specified a PDF viewer in the
GUI configuration (Misc$\rightarrow$Configuration).

\section{Tutorial}

This section gives a short tutorial to usage of \memorax.

Start the GUI.

\noindent
\begin{verbatim}
$ memorax-gui &
\end{verbatim}

The GUI window shows an example \rmm\ program that can be analysed, an
output area showing the version of the GUI, and a number of controls
that allow the user to select a command and options.

\paragraph{Reachability analysis}
First, let us analyse the reachability of the forbidden states in the
example program: Select the command ``Reachability'', the abstraction
``SB'' and the verbosity ``Messages''. Press the ``Run'' button to
start the analysis.

In case the GUI is unable to find the \memorax\ CLI, then you will receive
an error message:

\noindent
\begin{verbatim}
Failed to start subprocess (...)
[Errno 2] No such file or directory
Failed to terminate subprocess.
Interrupted
\end{verbatim}

If so, enter Misc$\rightarrow$Configuration and setup the correct path
to where you have installed the \memorax\ CLI.

If the GUI finds the CLI, you should instead receive a screenful of
text describing the result. The most important part is the last
section. It tells you that the forbidden states are reachable when the
example program is executed under the TSO memory model. I.e. that the
program is unsafe.

\noindent
\begin{verbatim}
Reachability analysis results:
  Reachable:             Yes
  Generated constraints: 500
  Size of visited set:   216
  Time consumption:      0 s
\end{verbatim}

You will also receive a ``witness trace'' showing \emph{how} the
forbidden states can be reached in the SB semantics.

\paragraph{Fence inference}
Now, let us see how \memorax\ can be used to automatically infer the
fences that are necessary to make the example program safe. Select the
command ``Fence insertion''. Keep the abstraction ``SB'' and verbosity
``Messages''. Press the ``Run'' button. 

If all goes well you should receive an output like this:

\noindent
\begin{verbatim}
$ /path/to/memorax fencins --json -v --abstraction sb
Currently examining fence set:
  (No fences)

Reachability analysis results:
  Reachable:             Yes
  Generated constraints: 500
  Size of visited set:   216
  Time consumption:      0.01 s

Cycles found in trace:
TsoCycle (complete):
  P0: update(var:0, P0)
  L14 P0: read: var:1 = 0
  L22 P1: locked{ write: var:1 := 1 }
  L23 P1: read: var:0 = 0

Currently examining fence set:
  L13 P0: write: x := 1
  L22 P1: write: y := 1

Reachability analysis results:
  Reachable:             No
  Generated constraints: 86
  Size of visited set:   39
  Time consumption:      0 s

Found 1 fence set:
Fence set #0:
  L13 P0: write: x := 1
  L22 P1: write: y := 1

Total time to insert fences: 0.01s.
\end{verbatim}

Reading it from top to bottom, it tells us the following:

\noindent
\begin{verbatim}
Currently examining fence set:
  (No fences)
\end{verbatim}

The inference procedure starts without any inserted memory fences.

\noindent
\begin{verbatim}
Reachability analysis results:
  Reachable:             Yes
  Generated constraints: 500
  Size of visited set:   216
  Time consumption:      0.01 s
\end{verbatim}

Without any memory fences, the forbidden states are reachable. At the
verbosity level ``Messages'', the witness traces are omitted. If you
want to see the traces, use e.g. ``Debug'' instead.

\noindent
\begin{verbatim}
Cycles found in trace:
TsoCycle (complete):
  P0: update(var:0, P0)
  L14 P0: read: var:1 = 0
  L22 P1: locked{ write: var:1 := 1 }
  L23 P1: read: var:0 = 0

Currently examining fence set:
  L13 P0: write: x := 1
  L22 P1: write: y := 1
\end{verbatim}

The inference procedure analyses the witness trace, and concludes that
in order to prevent the example program from reaching the forbidden
states by such an execution, two memory fences are necessary. The
memory fences are ``{\tt L13 P0: write: x := 1}'' and ``{\tt L22 P1:
  write: y := 1}''. This notation should be interpreted as follows:
{\tt L13 P0: write: x := 1} is the writing instruction of process 0
that occurs at line 13 in the code. The corresponding fence, which is
suggested by the inference procedure, should be placed immediately
after this writing transition. In the \rmm\ language, inserting the
fence is done by changing {\tt write: x := 1} into {\tt locked write:
  x := 1} in the code.

\noindent
\begin{verbatim}
Reachability analysis results:
  Reachable:             No
  Generated constraints: 86
  Size of visited set:   39
  Time consumption:      0 s
\end{verbatim}

The inference procedure attempts another reachability analysis, now
with the two new fences inserted. This time it turns out that the
forbidden states are not reachable, and the current fence set is
sufficient for safety.

\noindent
\begin{verbatim}
Found 1 fence set:
Fence set #0:
  L13 P0: write: x := 1
  L22 P1: write: y := 1
\end{verbatim}

The inference procedure terminates, telling us that it detected
exactly one minimal and sufficient set of memory fences: 
\begin{displaymath}
\{\texttt{L13 P0: write: x := 1},\texttt{L22 P1: write: y := 1}\}
\end{displaymath}
Hovering the mouse over the fence set will highlight the
corresponding write instructions in the code area. Clicking the fence
set will center the code over the highlighted instructions.

\paragraph{Adding the fences}
Let us manually insert the fences, and then try again. Rewrite the
code by adding ``{\tt locked}'' in two places as shown below:

\begin{center}
\small{\tt
\begin{tabular}{l|l}
/* An example code */\\
\\
forbidden\\
  CS CS\\
\\
data\\
  x = 0 : [0:1]\\
  y = 0 : [0:1]\\
\\
process                  & process\\
text                     & text\\
L0:                      & L0:\\
  \fbox{locked} write: x := 1;  & \fbox{locked} write: y := 1;\\
  read: y = 0;           & read: x = 0;\\
CS:                      & CS:\\
  write: x := 0;         & write: y := 0;\\
  goto L0                & goto L0 \\
\end{tabular}}
\end{center}

The previous fence inference result told us that this new version of
the example code should be safe. To satisfy our curiosity and to see
what it looks like when we run the fence inference procedure on an
already safe program, let us try to run the fence insertion command
again. We get the following result:

\noindent
\begin{verbatim}
$ /path/to/memorax/build/bin/memorax fencins --json -v --abstraction sb
Currently examining fence set:
  (No fences)

Reachability analysis results:
  Reachable:             No
  Generated constraints: 86
  Size of visited set:   39
  Time consumption:      0 s

Found 1 fence set:
Fence set #0:
  (No fences)

Total time to insert fences: 0s.
\end{verbatim}

It tells us that the fence inference procedure starts with no fences
(no fences except the ones that are explicitly part of the
program). It runs the reachability analysis and finds that the
forbidden states are not reachable; the program is safe. \memorax\ 
concludes by telling us that it found exactly one memory fence set
that is necessary and sufficient: the empty set. I.e. as expected, the
program is safe and requires no additional fences.

\section{The \rmm\ language}\label{sec:rmm:language}

The \rmm\ language allows to model a parallel program and specify safety
properties that should hold.

The sometimes assembly-like syntax of the \rmm\ language is motivated by
the necessity, when analysing programs under relaxed memory models, of
unambiguously specifying the order of memory accesses, and whether
variables are stored in memory or in registers. Note that for
conventional programming languages, such as e.g. C, the memory access
ordering, register allocations, reuse of common sub-expressions and the
like depend on the compiler (and compiler switches). Therefore, when
trying to verify an implementation written in a high level language,
it may be necessary to examine the machine code after compilation, or
use inline assembly, to be certain that the model, written in \rmm, and
the compiled program, written in some high level language, correspond.

This section will start by introducing the \rmm\ language by giving an
example together with explanation. Then we continue by explaining the
abstract machine on which an \rmm\ program runs. We introduce control
structures and informally explain about instructions under the
Sequentially Consistent memory model. Finally we give formally and
informally, the semantics of all instructions under the TSO memory
model.

\subsection{Machine Model and Memory Addressing}\label{sec:abstmachinememaddr}

This section describes the abstract machine on which an \rmm\ program is
executed. 

\newcommand{\vars}{\mathcal{X}} % The set of memory locations
\newcommand{\procs}{\mathcal{P}} % The set of processes
\newcommand{\automata}{A} % The set of process automata
\newcommand{\pautomaton}[1]{A(#1)} % \pautomaton{p} is the automaton
                                    % of process p
\newcommand{\regs}{\mathcal{R}} % \regs is the set of register sets
\newcommand{\pregs}[1]{\mathcal{R}(#1)} % \pregs{p} is the registers
                                         % of process p
\newcommand{\pcstates}[1]{\mathcal{Q}_{#1}} % \pcstates{p} is the set
                                            % of control states of
                                            % \pautomaton{p}
\newcommand{\ptransitions}[1]{\Delta_{#1}} % \ptransitions{p} is the
                                           % set of transitions of
                                           % \pautomaton{p} 


A \emph{machine} $(\procs,\automata,\regs,\vars)$ consists of a memory
with (shared) memory locations $\vars$, and a set of processes
$\procs$ executing in parallel. Each process $p\in\procs$ has a unique
process identifier $pid(p) \in \{0,\cdots,|\procs|-1\}$. We will
subsequently use $p$ and $pid(p)$ interchangingly where there is no
danger of confusion. 

\paragraph{Automata}
Each process $p$ is equipped with an automaton $\pautomaton{p} =
(\pcstates{p},\ptransitions{p})$ describing the program executed by
$p$. The set $\pcstates{p} = \{0,\cdots\}$ is the set of control
states of the automaton. The set $\ptransitions{p}$ is the set of
transitions of the automaton. A transition $(q_0,instr,q_1)$ consists
of a source control state $q_0$, a target control state $q_1$ and an
\rmm\ \emph{instruction}. Instructions will be defined and given
semantics in later sections.

\paragraph{Registers} 
Each process $p$ has a set of registers $\pregs{p}$. Registers
$r\in\pregs{p}$ hold integer values and can only be accessed by the
owning process $p$. Registers are not affected by memory model
relaxations since they are not located in memory (and also are
private). The differences between registers and memory locations are
summarised in \cref{tbl:memregdiff}.

\begin{table}[ht]
  \small{
  \begin{center}
    \begin{tabular}{|l|c|c|}
      \cline{2-3}
      \multicolumn{1}{c|}{} & Memory locations & Registers\\
      \hline
      Integer values & Yes & Yes\\
      \hline
      Accessible by & All processes & Owning process\\
      \hline
      In arith. expr. & No & Yes\\
      \hline
      Write & {\tt write}-instruction & 
      \begin{tabular}{@{}c@{}}
      Assignment instruction\\
      (E.g. {\tt \$r0 := \$r1 + 1})\\
      or assigning read\\
      (E.g. {\tt read: \$r0 := x})\\
      \end{tabular}\\
      \hline
      Read & 
      \begin{tabular}{@{}c@{}}
        {\tt read}-instruction \\
        (assigning read or \\
        asserting read)\\
      \end{tabular} &
      \begin{tabular}{@{}c@{}}
        Use in arithmetic expression\\
        (E.g. {\tt \$r1 + 1})\\
      \end{tabular}\\
      \hline
      Names & 
      \begin{tabular}{@{}c@{}}
        Alphanumerical or\\
        alphanumerical followed\\
        by process specifier\\
      \end{tabular} & {\tt \$} followed by alphanumerical\\
      \hline
    \end{tabular}
  \end{center}
  }
  \caption{Differences between memory locations and
    registers.}\label{tbl:memregdiff}
\end{table}

\paragraph{Integer Domains}
Each memory location $v$ and each register $r$ has an associated
domain $domain(v), domain(r) \subseteq\mathbb{Z}$. The domain is
either the (infinite) set of integers $\mathbb{Z}$, or a finite
interval $\{i,i+1,\cdots,j-1,j\}$. 

\paragraph{Memory Addressing}
Memory locations in $\vars$ are of two kinds: \emph{local} and
\emph{global}. Both kinds are accessible for reading and writing by
all processes, and there is no difference between the two kinds
regarding memory model relaxation. The difference is purely in how
they are addressed. A global memory location $v$ has an alphanumerical
name $n$, and is addressed by all processes by precisely that name. A
local memory location $v$ has an alphanumerical name $n$, but is also
associated with one particular process $owner(v)\in\procs$. When a
process $p$ accesses the local variable $v$, it should use the name
$n$\verb+[+$spec$\verb+]+ where $spec$ depends on $owner(v)$ and on
$p$ as described in \cref{tbl:locvarnames}. For example: suppose that
process 2 should access a local variable $v$ by the name \verb+x+. If
$owner(v) = 1$ then the correct address of $v$ for process 2 would be
\verb+x[1]+. If $owner(v) = 2$ then \verb+x[my]+ would be the address,
and if $owner(v) = 5$ then \verb+x[4]+ would be the correct address.

\begin{table}[ht]
  \begin{center}
    \begin{tabular}{|l|l|}
      \hline
      condition & $spec$\\
      \hline
      $owner(v) < p$ & $owner(v)$\\
      $owner(v) = p$ & \verb+my+\\
      $owner(v) > p$ & $owner(v)-1$\\
      \hline
    \end{tabular}
  \end{center}
  \caption{Process $p$, when accessing variable $v$ declared locally
    in process $owner(v)$, should use the name
    $v${\tt [}$spec${\tt ]}.}
  \label{tbl:locvarnames}
\end{table}

\paragraph{Pointers}
As a third way of addressing memory, pointer expressions are allowed
in \rmm. The syntax is {\tt [$e$]}, where $e$ is an arithmetic
expression over literal integers and register values. A pointer {\tt
  [$e$]} occurring in a statement $stmt(${\tt [$e$]}$)$ in \rmm\ is
really syntactic sugar for the following composed statement:

\noindent
{\tt either\{}\\
{\tt \rule{5pt}{0pt} assume: $e$ = 0; $stmt(v_0)$}\\
{\tt or}\\
{\tt \rule{5pt}{0pt} $\cdots$}\\
{\tt or}\\
{\tt \rule{5pt}{0pt} assume: $e$ = $n$; $stmt(v_n)$}\\
{\tt \}}\\

Here $v_0 \cdots v_n$ are all {\em global} memory locations in the
program, in the order they were declared. Local memory locations
cannot be accessed through pointers.

\subsection{An Example}

\begin{figure}[ht]
\begin{minipage}[b]{0.5\linewidth}
\small{
\begin{verbatim}
 1: /* Dijkstra's lock */
   
 2: forbidden
 3:   CS CS
   
 4: data
 5:   turn = * : [0:1]
   
 6: process
 7: data
 8:   flag = 0 : [0:2]
 9: registers
10:   $flag = * : [0:2]
11:   $turn = * : [0:1]
12: text
13:   START:
14:   write: flag[my] := 1;
15:   read: $turn := turn;
16:   while $turn != 0 do{
17:     read: $flag := flag[0];
18:     if $flag = 0 then
19:       write: turn := 0;
20:     read: $turn := turn
21:   };
22:   write: flag[my] := 2;
23:   read: $flag := flag[0];
24:   if $flag = 2 then
25:     goto START;
26:   CS:
27:   write: flag[my] := 0;
28:   goto START
\end{verbatim}
}
\end{minipage}
\hspace{0pt}
\begin{minipage}[b]{0.45\linewidth}
\small{
\begin{verbatim}
29: process
30: data
31:   flag = 0 : [0:2]
32: registers
33:   $flag = * : [0:2]
34:   $turn = * : [0:1]
35: text
36:   START:
37:   write: flag[my] := 1;
38:   read: $turn := turn;
39:   while $turn != 1 do{
40:     read: $flag := flag[0];
41:     if $flag = 0 then
42:       write: turn := 1;
43:     read: $turn := turn
44:   };
45:   write: flag[my] := 2;
46:   read: $flag := flag[0];
47:   if $flag = 2 then
48:     goto START;
49:   CS:
50:   write: flag[my] := 0;
51:   goto START
\end{verbatim}
}
\end{minipage}
\caption{\rmm\ model of two processes using Dijkstra's mutual exclusion protocol\cite{LP93}}\label{fig:code:dijkstra.rmm}
\end{figure}

\Cref{fig:code:dijkstra.rmm} shows an \rmm\ model of two processes using
Dijkstra's mutual exclusion protocol. We will explain the format line
by line.

The first line is a comment. Everything starting with \verb+/*+
continuing until \verb+*/+ is ignored by the parser.

Lines 2 and 3 declare the safety property. Every \rmm\ file must start
with such a declaration. The word \verb+forbidden+ is a reserved
word. Line 3 tells us that the declared safety property states that at
no time may simultaneously process 0 be at its control state labelled
\verb+CS+ (line 26) and process 1 be at its control state labelled
\verb+CS+ (line 49). The label names \verb+CS+ and \verb+CS+ are
coincidentally the same, but refer to different processes and hence
different control states. Additional lines like line 3 can be added
provided that they are separated by semi-colons. Below we have added
the safety properties that none of the processes may enter its
critical section (label CS) while the other process is at its initial
control state (label START).

\begin{verbatim}
forbidden
  CS CS ; 
  START CS ;
  CS START
\end{verbatim}

Lines 4 and 5 declare a memory location called \verb+turn+. The word
\verb+data+ is a reserved word. After the word \verb+data+ comes a
list of memory location declarations. The declaration
\verb+turn = * : [0:1]+ starts with the name of the memory location.
Then states (\verb+= *+) that it may initially have any value in its
domain. The last part (\verb+: [0:1]+) specifies that the domain of
the memory location is all integers from and including 0 up to and
including 1. Below we have extended the declaration section to also
declare a variable \verb+x+ with domain $\mathbb{Z}$ and initial
value 0, and a variable \verb+y+ with the default domain (which is
also $\mathbb{Z}$) and unspecified initial value.

\begin{verbatim}
data
  turn = * : [0:1]
  x = 0 : Z
  y = *
\end{verbatim}

Lines 6 to 28 declare process 0. Lines 29 to 51 similarly declare
process 1, and will not be separately explained. The word
\verb+process+ on line 6 is a reserved word and informs us that a
process declaration begins. The process declaration has three parts:
data declaration (optional), register declaration (optional) and text
declaration (mandatory).

Lines 7 and 8 is the data declaration for process 0. It declares a
memory location named flag, with domain $\{0,1,2\}$ and initial value
0. This memory location is like the memory location \verb+turn+ that
we declared earlier, in that it is accessible for both reading and
writing to all processes and in that it is affected by the memory
model. The only difference between memory locations declared at the
top level (\emph{global} memory locations) and memory locations
declared inside a process declaration (\emph{local} memory locations)
is the naming. In order to access a global memory location, a process
will use its name as it is. A local memory location $var$ is accessed
by its name and a specifier: $var$\verb+[+$spec$\verb+]+. The correct
way of addressing local memory locations is described in
\cref{sec:abstmachinememaddr} and in particular in
\cref{tbl:locvarnames}.

Lines 9 to 11 declare the registers of process 0. Registers are
similar to memory locations. They correspond to processor registers,
so they are accessible only to the process owning them, and they are
not affected by the memory model. In \rmm, registers have
alphanumerical names preceded by a single \verb+$+ character.

The word \verb+text+ on line 12 informs that the program code begins. 

The program code is a semi-colon separated sequence of
statements. Each statement is optionally preceded by a process-unique
label and a colon.

Line 13 declares a label \verb+START+ that identifies the control
state immediately before execution of the first instruction.

Line 14, 19, 22 and 27 are memory writes. A value computed by
arithmetic operations on literal integers and on values in registers
is assigned to a memory location. In this case literal integers 0, 1
and 2 are stored in the global memory location \verb+turn+ and the
local memory location \verb+flag+ of process 0.

Line 15, 17, 20 and 23 are memory reads. The value in a memory
location is loaded into a register. In the case of line 17, the value
in the local variable \verb+flag+ of process 1 (see
\cref{tbl:locvarnames}) is loaded into the register \verb+$flag+ of
process 0.

Lines 16 to 21 is a while-loop. The loop condition works on literal
integers and values in registers. 

Lines 18-19 and 24-25 are if-statements. The if-condition works on
literal integers and values in registers. If-statements may optionally
have an else-clause.

Lines 25 and 28 are goto-statements. A goto-statement \verb+goto +$LBL$
immediately redirects the control flow to the control state labelled
by $LBL$.

\subsection{Semantics}\label{sec:semantics}
\subsubsection{Statements}

In the \rmm\ language a process's automaton is defined by a statement
that follows the keyword \verb+text+ in the process
declaration. Statements come in two kinds: \emph{instructions} and
\emph{control structures}. 

An \emph{instruction} is a statement that can be attached to a
transition in an automaton, and be executed atomically. An instruction
corresponds loosely to a machine language instruction on an actual
piece of hardware. But there are differences: Some composed operations
that would require many machine language instructions can be performed
in a single \rmm\ instruction. This is the case for local operations,
such as evaluation of arithmetic expressions over private registers,
where the non-atomicity of the operation on actual hardware is not
observable. Furthermore, some instructions on hardware architectures,
such as e.g. un-LOCK'd INC on Intel x86, appear as a single
instruction in the machine language but executes equivalently to
multiple subsequent and non-atomic memory accesses. Such instructions
are not included in \rmm.

A \emph{control structure} is a statement that affects the structure
of the process automaton. Control structures themselves cannot occur
as labels for individual transitions, but control structures may
contain instructions and define a sub-automaton with transitions
labelled by those instructions.

\Cref{tbl:allstatements} lists all types of \rmm\ statements.

\begin{table}
  \begin{center}
  \small{
  \begin{tabular}{|ll|}
    \hline
    \hline
    \multicolumn{2}{|c|}{Instructions}\\
    \hline
    \hline
    Name & Example \\
    \hline
    Nop & {\tt nop}\\ & \\
    Assignment & {\tt \$reg := 42}\\ & \\
    Assume & {\tt assume: \$r0 = 0 \&\& \$r1 > 2}\\ & \\
    Asserting read & {\tt read: x = 3}\\ & \\
    Assigning read & {\tt read: \$reg := x}\\ & \\
    Write & {\tt write: x := \$r0 + \$r1 - 1}\\ & \\
    Locked block & {\tt 
      \begin{tabular}{@{}l@{}}
        locked\verb+{+\\
        \rule{5pt}{0pt} read: x = 0;\\
        \rule{5pt}{0pt} write: x := 1\\
        \verb+}+\\
      \end{tabular}
    }\\ & \\
    (Locked write) & {\tt locked write: x := 1}\\ & \\
    (Compare \& Swap) & {\tt cas(x,2,13)}\\ & \\
    \hline
    \hline
    \multicolumn{2}{|c|}{Control statements}\\
    \hline
    \hline
    Name & Example \\
    \hline
    Goto & {\tt goto LBL}\\ & \\
    Sequence & {\tt 
      \begin{tabular}{@{}l@{}}
        \verb+{+\\
        \rule{5pt}{0pt} read: \$r0 := x;\\
        \rule{5pt}{0pt} \$r1 := \$r0 + 1;\\
        \rule{5pt}{0pt} write: x := \$r1\\
        \verb+}+\\
      \end{tabular}
    }\\ & \\
    If-statement & 
    {\tt 
    \begin{tabular}{@{}l@{}}
    if \$r0 = 0 then \verb+{+\\
    \rule{5pt}{0pt} \$r1 := 1; \\
    \rule{5pt}{0pt} \$r2 := 20 \\
    \verb+}+ else\\
    \rule{5pt}{0pt} goto L0
    \end{tabular}}\\ & \\
    While-statement &
    {\tt
    \begin{tabular}{@{}l@{}}
      while \$reg > 0 do\\
      \rule{5pt}{0pt} read: \$reg := x\\
    \end{tabular}}\\ & \\
    Either-statement &
    {\tt
    \begin{tabular}{@{}l@{}}
      either\verb+{+\\
      \rule{5pt}{0pt} read: v = 0\\
      or\\
      \rule{5pt}{0pt} read: v = 1;\\
      \rule{5pt}{0pt} write: w := 1\\
      \verb+}+\\
    \end{tabular}
    }\\ & \\
    \hline
  \end{tabular}
  }
  \end{center}
  \caption{\rmm\ statements}\label{tbl:allstatements}
\end{table}

\subsubsection{Arithmetical and Boolean Expressions}

Some statements make use of arithmetical or boolean expressions. An
arithmetical expression may contain registers, integer literals,
addition, subtraction, unary minus and parentheses. A boolean
expression may contain the boolean literals \verb+true+ and
\verb+false+, conjunction (\verb+&&+), disjunction (\verb+||+),
negation (\verb+not+), parentheses (\verb+[]+ is used for boolean
expressions as opposed to \verb+()+ for arithmetical) and comparison
of arithmetical expressions by the following comparison functions:
\verb+=+, \verb+!=+, \verb+<+, \verb+>+. The
expressions are interpreted in the obvious way. Note that memory
locations cannot be used in expressions!

\subsubsection{Instructions Informally}

This section gives an informal description of the semantics of \rmm\ 
instructions under the SC memory model.

In the below, we use the following conventions: Registers are named
$reg$, $reg'$ etc. Registers used in an instruction always refer to
registers owned by the process that executes the
instruction. Arithmetical expressions are named $expr$, $expr'$
etc. Boolean expressions are named $bexpr$, $bexpr'$ etc. Memory
locations are named $v$, $v'$ etc. That an instruction is
\emph{enabled} means that it can be executed. An instruction that is
not enabled is blocking.

\paragraph{Nop} \verb+nop+

This instruction is always enabled, and has no effect when executed.

\paragraph{Assignment} $reg$\verb+ := +$expr$

Evaluates the expression $expr$ and stores the result in the register
$reg$. The instruction is enabled precisely when the valuation of
$expr$ is within the domain of $reg$.

\paragraph{Assume} \verb+assume: +$bexpr$

Is enabled precisely when $bexpr$ evaluates to true. The instruction
has no effect when executed.

\paragraph{Asserting Read} \verb+read: +$v$\verb+ = +$expr$

Is enabled when the value of memory location $v$ in memory is the same
as the value to which $expr$ evaluates. The instruction has no effect
when executed.

\paragraph{Assigning Read} \verb+read: +$reg$\verb+ := +$v$

Reads the value of memory location $v$ from memory and stores the
value in register $reg$. The instruction is enabled precisely when the
value of $v$ in memory is within the domain of $reg$.

\paragraph{Write} \verb+write: +$v$\verb+ := +$expr$

Evaluates the expression $expr$ and writes the result to memory
location $v$. Enabled when the value of $expr$ is in the domain of
$v$.

\paragraph{Locked Block} \mbox{}

\noindent
\begin{tabular}{@{}l@{}}
  \verb+locked{+\\
  \rule{5pt}{0pt}$sl_0$\\
  \verb+or+\\
  \rule{5pt}{0pt}$\cdots$\\
  \verb+or+\\
  \rule{5pt}{0pt}$sl_n$\\
  \verb+}+\\
\end{tabular}

Here $sl_i$ for all $0 \leq i \leq n$ is a semi-colon separated
sequence of instructions. The sequence $sl_i$ is said to be enabled if
it is possible to execute its constituent instructions in order
without blocking and without context-switching. When the locked block
is executed, any one single enabled sequence $sl_i$ is picked, and the
instructions of $sl_i$ are executed in order atomically. The locked
block is enabled when there is at least one sequence $sl_i$ that is
enabled.

{\bf Important:} The locked block is a powerful construction meant to
enable modellers to model the occasional more obscure machine
instructions that may occur on their hardware. Its TSO semantics are
quite complicated, and improper use is easy. It is therefore
recommended to not explicitly use locked blocks unless absolutely
necessary. It is recommended to limit use of locked blocks to implicit
use by means of the two instructions \emph{locked write} and
\emph{Compare \& Swap}.

\paragraph{Locked Write} \verb+locked write: +$v$\verb+ := +$expr$

Under SC semantics, a locked write is equivalent to an ordinary
write. The locked write \verb+locked write: +$v$\verb+ := +$expr$ is
syntactic sugar for \\\verb+locked{ write: +$v$\verb+ := +$expr$\verb+}+

\paragraph{Compare \& Swap (CAS)} \verb+cas(+$v$\verb+,+$expr$\verb+,+$expr'$\verb+)+

The compare and swap instruction is enabled precisely when the value
of $v$ in memory is equal to the value of $expr$, and $expr'$ evaluates
to a value within the domain of $v$. Executing the compare and swap
instruction will store the value of $expr'$ in memory location
$v$. The instruction
\verb+cas(+$v$\verb+,+$expr$\verb+,+$expr'$\verb+)+ is syntactic sugar
for
\verb+locked{ read: +$v$\verb+ = +$expr$\verb+; write: +$v$\verb+ := +$expr'$\verb+}+

\subsubsection{Control Structures}

\newcommand{\tikzq}[3]{\node (#1) at (#2) [circle,draw=black,inner sep=0pt,minimum size=25pt] {$#3$}}

This section describes how control structures in the \rmm\ language are
used to shape the process automata of a machine. This is done by
describing the sub-automata corresponding to each type of control
structure.

Below we use the following conventions: The described sub-automaton is
a part of the automaton $\pautomaton{p}$ corresponding to process
$p$. The initial state of the sub-automaton is $q_{src}$. The control
state that corresponds to the position in the \rmm\ code immediately
after the control structure is named $q_{tgt}$.

\paragraph{Instructions} $s$

A single instruction $s$ translates into a transition
$(q_{src},s,q_{tgt})$.

\begin{center}
\small{
\begin{tikzpicture}
  \tikzq{qsrc}{0,0}{q_{src}};
  \tikzq{qtgt}{2,0}{q_{tgt}};
  \draw [->] (qsrc) -- node [above] {$s$} (qtgt);
  \draw[draw=black,rounded corners] ($(current bounding box.north west) + (-0.5,0.5)$) rectangle
  ($(current bounding box.south east) + (0.5,-0.5)$);
\end{tikzpicture}
}
\end{center}

\paragraph{Goto} \verb+goto +$LBL$

Here $LBL$ is a label attached to some control state $q_{LBL}$ in
$\pcstates{p}$. The goto statement translates into a transition
$(q_{src},\verb+nop+,q_{LBL})$.

\begin{center}
\small{
\begin{tikzpicture}
  \tikzq{qsrc}{0,0}{q_{src}};
  \tikzq{qlbl}{1.5,-1.5}{q_{LBL}};
  \tikzq{qtgt}{2,0}{q_{tgt}};
  \draw[->] (qsrc) -> node [above] {\tt nop} (qlbl);
  \draw[draw=black,rounded corners] ($(current bounding box.north west) + (-0.5,0.5)$) rectangle
  ($(current bounding box.south east) + (0.5,-0.5)$);
\end{tikzpicture}
}
\end{center}


\paragraph{Sequence} \verb+{+$s_0$\verb+;+$\cdots$\verb+;+$s_n$\verb+}+

The sequence construct arranges its constituent sub-statements,
unsurprisingly, in a sequence from $q_{src}$ to $q_{tgt}$, as shown in
the diagram below.

\begin{center}
\small{
\begin{tikzpicture}
  \tikzq{qsrc}{0,0}{q_{src}};
  \tikzq{q1}{2,0}{};
  \tikzq{q2}{4,0}{};
  \tikzq{qtgt}{6,0}{q_{tgt}};
  \draw[->] (qsrc) -- node [above] {$s_0$} (q1);
  \draw[->] (q1) -- node [above] {$\cdots$} (q2);
  \draw[->] (q2) -- node [above] {$s_n$} (qtgt);

  \draw[draw=black,rounded corners] ($(current bounding box.north west) + (-0.5,0.5)$) rectangle
  ($(current bounding box.south east) + (0.5,-0.5)$);
\end{tikzpicture}
}
\end{center}

\paragraph{If-statement} \verb+if +$bexpr$\verb+ then +$s_{then}$\verb+ else +$s_{else}$

An if-statement branches the automaton into two branches where the
initial assume transitions ensure that only one branch can be taken at
any one time.

\begin{center}
\small{
\begin{tikzpicture}
  \tikzq{qsrc}{0,0}{q_{src}};
  \tikzq{q1}{3,1}{};
  \tikzq{q2}{3,-1}{};
  \tikzq{qtgt}{6,0}{q_{tgt}};
  \draw[->,out=70,in=180] (qsrc) to node [above] {{\tt assume: }$bexpr$} (q1);
  \draw[->,out=290,in=180] (qsrc) to node [below] {{\tt assume: not }$bexpr$} (q2);
  \draw[->,out=0,in=110] (q1) to node [above] {$s_{then}$} (qtgt);
  \draw[->,out=0,in=250] (q2) to node [below] {$s_{else}$} (qtgt);

  \draw[draw=black,rounded corners] ($(current bounding box.north west) + (-0.5,0.5)$) rectangle
  ($(current bounding box.south east) + (0.5,-0.5)$);
\end{tikzpicture}
}
\end{center}

\paragraph{While-statement} \verb+while +$bexpr$\verb+ do +$s$

A while-statement translates to a loop that will be taken as long as
$bexpr$ evaluates to true, but no longer.

\begin{center}
\small{
\begin{tikzpicture}
  \tikzq{qsrc}{0,0}{q_{src}};
  \tikzq{q1}{0,2}{};
  \tikzq{qtgt}{3,0}{q_{tgt}};
  \draw[->,out=135,in=225] (qsrc) to node [left] {{\tt assume: }$bexpr$} (q1);
  \draw[->,out=315,in=45] (q1) to node [right] {$s$} (qsrc);
  \draw[->,out=335,in=205] (qsrc) to node [below] {{\tt assume: not }$bexpr$} (qtgt);

  \draw[draw=black,rounded corners] ($(current bounding box.north west) + (-0.5,0.5)$) rectangle
  ($(current bounding box.south east) + (0.5,-0.5)$);
\end{tikzpicture}
}
\end{center}

\paragraph{Either-statement} 
\verb+either{+$s_0^0$\verb+;+$\cdots$\verb+;+$s_0^m$\verb+ or +$\cdots$\verb+ or +$s_n^0$\verb+;+$\cdots$\verb+;+$s_n^l$\verb+}+

An either statement is a non-deterministic choice. It translates into
multiple branches with no attached guards (except for what may occur
in the constituents $s_i^j$).


\begin{center}
\small{
\begin{tikzpicture}
  \tikzq{qsrc}{0,0}{q_{src}};
  \tikzq{q1}{2,2}{};
  \node (vd) at (2,0) [] {$\vdots$};
  \tikzq{q2}{2,-2}{};
  \tikzq{q3}{4,2}{};
  \node (vd2) at (4,0) [] {$\vdots$};
  \tikzq{q4}{4,-2}{};
  \tikzq{qtgt}{6,0}{q_{tgt}};

  \draw[->,out=70,in=180] (qsrc) to node [above] {$s_0^0$} (q1);
  \draw[->,out=290,in=180] (qsrc) to node [below] {$s_n^0$} (q2);
  \draw[-] (qsrc) -- ($(qsrc) + (1,1)$);
  \draw[-] (qsrc) -- ($(qsrc) + (1,0)$);
  \draw[-] (qsrc) -- ($(qsrc) + (1,-1)$);

  \draw[->] (q1) -- node [above] {$\cdots$} (q3);
  \draw[->] (q2) -- node [below] {$\cdots$} (q4);

  \draw[->,out=0,in=110] (q3) to node [above] {$s_0^m$} (qtgt);
  \draw[->,out=0,in=250] (q4) to node [below] {$s_n^l$} (qtgt);
  \draw[->] ($(qtgt) + (-1,1)$) -- (qtgt);
  \draw[->] ($(qtgt) + (-1,0)$) -- (qtgt);
  \draw[->] ($(qtgt) + (-1,-1)$) -- (qtgt);

  \draw[draw=black,rounded corners] ($(current bounding box.north west) + (-0.5,0.5)$) rectangle
  ($(current bounding box.south east) + (0.5,-0.5)$);
\end{tikzpicture}
}
\end{center}

\subsubsection{TSO Semantics}

This section describes the TSO semantics of the instructions in the
\rmm\ language. This is done formally and informally in parallel.

\newcommand{\mem}{M}
\newcommand{\regval}{M_{reg}}
\newcommand{\pc}{pc}
\newcommand{\buf}{B}

A \emph{constraint} $(\mem,\regval,\pc,\buf)$ describes the
configuration of an abstract machine at a particular time. Each memory
location $v\in\vars$ has a particular value $\mem(v)\in
domain(v)$. Each register $r$ owned by each process $p$ has a
particular value $\regval(p)(r)\in domain(r)$. Each process $p$ is at
a particular control state $\pc(p)\in\pcstates{p}$.

Furthermore, each process $p$ is equipped with a FIFO buffer
\\$\buf(p) = \langle \buf(p)_0, \cdots, \buf(p)_n\rangle$ where $n =
len(\buf(p)) - 1$. More recently inserted elements have a lower
index. For all $0 \leq i \leq n$, the element $\buf(p)_i = (v,w) \in
\vars\times\mathbb{Z}$ is a pending write of process $p$ to memory
location $v$ with value $w\in domain(v)$. 

When a process executes a write to a memory location under TSO, it
does not immediately update the memory, but instead it enqueues the
write in its write buffer $\buf(p)$. Asynchronously, and without the
active participation of the process $p$ itself, the enqueued writes
will be pushed, one by one in the same order they were enqueued, to
memory and dequeued from the write buffer. The event of a write
reaching memory is called an \emph{update}.

In the below we describe the rules for process $p$ to perform a
transition $t$. For a function $f$, by $f[x:=v]$ we denote the
function $f'$ such that $f'(y) = f(y)$ if $y \neq x$ and $f'(x) =
v$. For a register valuation $\regval(p)$ and an arithmetic expression
$expr$ over registers from $\pregs(p)$, we let $\regval(p)[expr]$
denote the evaluation of $expr$ where each register $r$ in $expr$
evaluates to $\regval(p)(r)$. We define similarly $\regval(p)[bexpr]$
for a boolean expression $bexpr$. 

\newcommand{\sep}{\rule{10pt}{0pt}}

\paragraph{Nop} $t = (q_{src},\verb+nop+,q_{tgt})$

The nop instruction is the same under TSO semantics as under SC
semantics: It does nothing.

\begin{displaymath}
  \frac{
    \pc(p) = q_{src}
  }{(\mem,\regval,\pc,\buf) \rightarrow_t (\mem,\regval,\pc[p:=q_{tgt}],\buf)}
\end{displaymath}

\paragraph{Assignment} $t = (q_{src},reg\verb+ := +expr, q_{tgt})$

The assignment instruction is the same under TSO semantics as under SC
semantics: It evaluates $expr$ and assigns the result to the register
$reg$.

\begin{displaymath}
  \frac{
    \begin{array}{c}
      \pc(p) = q_{src} \sep 
      w \in domain(reg)\\
      \regval' = \regval[p:=M_p']\\
      \textrm{where}\\
      w = \regval(p)[expr]\\
      M_p' = \regval(p)[reg:=w]\\
    \end{array}
  }
       {(\mem,\regval,\pc,\buf) \rightarrow_t (\mem,\regval',\pc[p:=q_{tgt}],\buf)}
\end{displaymath}

\paragraph{Assume} $t = (q_{src},\verb+assume: +bexpr,q_{tgt})$

The assume instruction is the same under TSO semantics as under SC
semantics: It evaluates $bexpr$ and is enabled precisely when the
result is true.

\begin{displaymath}
  \frac{
    \pc(p) = q_{src} \sep
    \regval(p)[bexpr]
  }{(\mem,\regval,\pc,\buf) \rightarrow_t (\mem,\regval,\pc[p:=q_{tgt}],\buf)}
\end{displaymath}

\paragraph{Asserting Read} $t = (q_{src},\verb+read: +v\verb+ = +expr,q_{tgt})$

A read under TSO semantics will read the value of memory location $v$
from memory, provided that the buffer of $p$ does not contain any
write to $v$. If there is a write to $v$ in $\buf(p)$, then the value
of the newest such write in $\buf(p)$ is read. 

To formalise this, we define the function $read :
((\vars\mapsto\mathbb{Z})\times buffer) \mapsto \vars \mapsto
\mathbb{Z}$ as follows:

\begin{displaymath}
  \begin{array}{c}
  read (\mem,b) v = \\
  \\
  \left\{\begin{array}{ll}
  w  & \textrm{If for some $i\in\mathbb{Z}$}\sep
  \left(\begin{array}{c}
    b_i = (v,w)\\
    \textrm{and}\\
    \neg\exists 0 \leq j < i, w'\in\mathbb{Z} . b_j = (v,w')
  \end{array}\right)\\
  \mem(v) & \textrm{Otherwise}\\
  \end{array}\right.\\
  \end{array}
\end{displaymath}

Now we can define the transition rule:

\begin{displaymath}
  \frac{
    \pc(p) = q_{src} \sep
    read (\mem,\buf(p)) v = \regval(p)[expr]
  }{(\mem,\regval,\pc,\buf) \rightarrow_t (\mem,\regval,\pc[p:=q_{tgt}],\buf)}
\end{displaymath}

\paragraph{Assigning Read} $t = (q_{src},\verb+read: +reg\verb+ := +v,q_{tgt})$

An assigning read, reads the value of $v$ from memory or from
$\buf(p)$ in the same manner as an asserting read, but then assigns
the read value to the register $reg$.

\begin{displaymath}
  \frac{
    \begin{array}{c}
      \pc(p) = q_{src} \sep 
      w \in domain(reg)\\
      \regval' = \regval[p:=M_p']\\
      \textrm{where}\\
      w = read (\mem,\buf(p)) v\\
      M_p' = \regval(p)[reg:=w]\\
    \end{array}
  }
       {(\mem,\regval,\pc,\buf) \rightarrow_t (\mem,\regval',\pc[p:=q_{tgt}],\buf)}
\end{displaymath}

\paragraph{Write} $t = (q_{src},\verb+write: +v\verb+ := +expr,q_{tgt})$

A write instruction evaluates the value $w$ of the expression $expr$,
and enqueues the write as $(v,w)$ in its buffer.

\begin{displaymath}
  \frac{
    \begin{array}{c}
      \pc(p) = q_{src} \sep w \in domain(v)\\
      \buf' = \buf[p := (v,w)\cdot\buf(p)]\\
      \textrm{where}\\
      w = \regval(p)[expr]\\
    \end{array}
  }{(\mem,\regval,\pc,\buf) \rightarrow_t (\mem,\regval,\pc[p:=q_{tgt}],\buf')}
\end{displaymath}

\paragraph{Update} $t = update_p$

An \emph{update} is not a transition in any process automaton. Instead
it is an event that may happen at any time the buffer of process $p$
is non-empty. When an update occurs, the oldest write $(v,w)$ in the
buffer of process $p$ is dequeued, and the value of variable $v$ is
assigned $w$.

\begin{displaymath}
  \frac{
    \begin{array}{c}
      \buf(p) = \langle \buf(p)_0,\cdots,\buf(p)_{n-1},(v,w)\rangle \\
      \buf' = \buf[p:=\langle \buf(p)_0,\cdots,\buf(p)_{n-1}\rangle] \\
    \end{array}
  }{(\mem,\regval,\pc,\buf) \rightarrow_t (\mem[v:=w],\regval,\pc,\buf')}
\end{displaymath}

\paragraph{Locked Write} $t = (q_{src},\verb+locked write: +v\verb+ := +expr,q_{tgt})$

A locked write acts as a write followed by a TSO fence. It requires
the buffer of process $p$ to be empty before it is executed. Then it
evaluates the value $w$ of $expr$ and writes $w$ directly to $v$ in
memory without enqueueing the write in the buffer.

\begin{displaymath}
  \frac{
    \begin{array}{c}
      \pc(p) = q_{src} \sep \buf(p) = \langle\rangle\\
      w \in domain(v)\\
      \textrm{where}\\
      w = \regval(p)[expr]\\
    \end{array}
  }{(\mem,\regval,\pc,\buf) \rightarrow_t (\mem[v:=w],\regval,\pc[p:=q_{tgt}],\buf)}
\end{displaymath}

The locked write $\verb+locked write: +v\verb+ := +expr$ is syntactic
sugar for \\\verb+locked{ write: +$v$\verb+ := +$expr$\verb+}+.

\paragraph{CAS} $t = (q_{src},\verb+cas(+v\verb+,+expr\verb+,+expr'\verb+)+,q_{tgt})$

A compare and swap instruction acts as a fence in that it requires the
buffer of process $p$ to be empty before it can be executed. It then
evaluates the values $w$ and $w'$ respectively for $expr$ and $expr'$,
compares the value of $v$ in memory with $w$, if the values are equal
then the value $w'$ is written to $v$ in memory, otherwise the
instruction blocks. A compare and swap instruction does not enqueue
any write to the buffer.

\begin{displaymath}
  \frac{
    \begin{array}{c}
      \pc(p) = q_{src} \sep \buf(p) = \langle\rangle\\
      \mem(v) = w \sep w' \in domain(v)\\
      \textrm{where}\\
      w = \regval(p)[expr] \sep w' = \regval(p)[expr']\\
    \end{array}
  }{(\mem,\regval,\pc,\buf) \rightarrow_t (\mem[v:=w'],\regval,\pc[p:=q_{tgt}],\buf)}
\end{displaymath}

\paragraph{Locked Block}
\verb+locked{+$s_0^0$\verb+;+$\cdots$\verb+;+$s_0^{m_0}$\verb+ or +$\cdots$\verb+ or +$s_n^0$\verb+;+$\cdots$\verb+;+$s_n^{m_n}$\verb+}+

A locked block acts as a fence iff there is some $s_i^j$ that is a
write. If so, then the whole locked block is enabled only if the
buffer of process $p$ is empty. 

When a locked block executes, it non-deterministically selects one
sequence $s_i^0\verb+;+\cdots\verb+;+s_i^{m_i}$ and executing atomically
all constituent instructions $s_i^j$ in order. When doing so each
constituent instruction executes as it normally does, except for
writes, which execute as locked writes.

In order to formalise the above, we first define the functions $is\_write$ and $contains\_write$:

\begin{displaymath}
  is\_write(s) = \left(\exists v\in\vars, w\in\mathbb{Z} . s = \verb+write: +v\verb+ := +w\right)
\end{displaymath}

\begin{displaymath}
  \begin{array}{c}
  contains\_write(s) = \\
  \\
  \left(
  \begin{array}{c}
    is\_write(s)\\
    \textrm{or}\\
    \left(
    \begin{array}{l}
    \exists s_0^0, \cdots, s_0^m, \cdots, s_n^0, \cdots, s_n^l, i, j . \\
    \sep s = \verb+locked{+s_0^0\verb+;+\cdots\verb+;+s_0^m\verb+ or +\cdots\verb+ or +s_n^0\verb+;+\cdots\verb+;+s_n^l\verb+}+\\
    \sep \textrm{and}\\
    \sep contains\_write(s_i^j)\\
    \end{array}\right)\\
  \end{array}
  \right)\\
  \end{array}
\end{displaymath}

Next we define the locked transition relation
$\rightarrow_s^{locked}$, for instructions $s$ by the following two
rules:

\begin{displaymath}
  \frac{
    \begin{array}{c}
      \neg is\_write(s) \\ 
      (\mem,\regval,\pc,\buf) \rightarrow_{(\pc(p),s,\pc(p))} (\mem',\regval',\pc',\buf')\\
    \end{array}
  }{(\mem,\regval,\pc,\buf) \rightarrow_s^{locked} (\mem',\regval',\pc',\buf')}
\end{displaymath}

\begin{displaymath}
  \frac{
    \begin{array}{c}
      is\_write(s) \\ 
      (\mem,\regval,\pc,\buf) \rightarrow_{(\pc(p),s,\pc(p))} (\mem',\regval',\pc',\buf')\\
      (\mem',\regval',\pc',\buf') \rightarrow_{update_p} (\mem'',\regval'',\pc'',\buf'')\\
    \end{array}
  }{(\mem,\regval,\pc,\buf) \rightarrow_s^{locked} (\mem'',\regval'',\pc'',\buf'')}
\end{displaymath}

Now we are ready to define the transition rule for the locked block. Let $s = \verb+locked{+s_0^0\verb+;+\cdots\verb+;+s_0^{m_0}\verb+ or +\cdots\verb+ or +s_n^0\verb+;+\cdots\verb+;+s_n^{m_n}\verb+}+$.

\begin{displaymath}
  \frac{
    \begin{array}{c}
      \pc(p) = q_{src}\\
      (contains\_write(s) \Rightarrow \buf(p) = \langle\rangle)\\
      \begin{array}{l}
        \exists 0 \leq i \leq n, c_0, \cdots, c_{m_i+1} .\\
        \sep \begin{array}{l}
          c_0 = (\mem,\regval,\pc,\buf) \\
          c_{m_i+1} = (\mem',\regval',\pc,\buf')\\
          \forall 0 \leq j \leq m_i . c_j \rightarrow_{s_i^j}^{locked} c_{j+1}\\
        \end{array}\\
      \end{array}\\
      \end{array}
  }{(\mem,\regval,\pc,\buf) \rightarrow_t (\mem',\regval',\pc[p:=q_{tgt}],\buf')}
\end{displaymath}

{\bf Important:} The locked block is a powerful construction meant to
enable modellers to model the occasional more obscure machine
instructions that may occur on their hardware. Its TSO semantics are
quite complicated, and improper use is easy. It is therefore
recommended to not explicitly use locked block unless absolutely
necessary. It is recommended to limit use of locked blocks to implicit
use by means of the two instructions \emph{locked write} and
\emph{Compare \& Swap}.

\subsection{Grammar}

\verbatiminput{grammar.simplified.txt}

\section{Abstractions}\label{sec:abstractions}
\subsection{PB - Bounded Buffers with Predicate Abstraction}

The PB abstraction is the implementation of \cite{AACLR12SAS}. It is
an over approximation of TSO.

The PB abstraction uses an over approximation of the TSO store
buffers. For a positive integer $k$, it stores the $k$ most recent
messages for each memory location and process. The information of
older messages is dropped.

Predicate abstraction is used to enable (infinite) integer domains for
memory locations and registers.

The reachability analysis is by backward state space exploration.

If CEGAR is used, then the value of $k$ as well as the set of
predicates for predicate abstraction is gradually refined. When CEGAR
is used, analysis and fence insertion with PB is sound, but not
complete. For fence insertion this means that any fence sets reported
by \memorax\ are sufficient and minimal for preventing reachability of
the forbidden states.

For details about the PB abstraction, see~\cite{AACLR12SAS}.

\subsection{SB - Single Buffer}

The SB abstraction is a reimplementation of
\cite{AACLR12}. Reachability analysis and fence insertion with SB is
both sound and complete.

The SB abstraction defines a program semantic which is equivalent to
TSO with regards to control state reachability. For the SB semantics,
the well-quasi ordering framework~\cite{ACJT96} is applicable and
provides a sound and complete reachability analysis.

The SB semantics replaces the TSO store buffers with a single, shared
store buffer. Each message in the single store buffer contains a
complete memory snapshot.

The reachability analysis is by backward state space exploration.

For details about the SB abstraction, see~\cite{AACLR12}.

\phantomsection
\addcontentsline{toc}{section}{References}
\bibliography{biblio}
\bibliographystyle{plain}

\pagebreak
\phantomsection
\addcontentsline{toc}{section}{GNU Free Documentation License}
\includepdf[pages=-]{fdl.pdf}

\end{document}

 - Problembeskrivning/introduktion
 - Installering
 - Beskrivning av användning av programmet
   * CLI
   * GUI
 - Beskrivning av språket
   * Grammatik
   * Kontrollstrukturer (beskrivning av hur de översätts till tillståndsmaskiner?)
   * TSO-semantik för instruktionerna
 - Beskrivning av abstraktionerna
   * SB
   * PB
